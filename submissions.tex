\documentclass[12pt]{article}
\setlength{\parindent}{0pt}
\usepackage[top=2cm, bottom=3cm, left=3cm, right=3cm]{geometry}
\geometry{a4paper}

\usepackage{titlesec}
\titleformat{\section}
    {\normalfont\bfseries}
    {\thesection}{1em}{}

\usepackage{fontspec,xltxtra,xunicode}
\defaultfontfeatures{Mapping=tex-text}
\setromanfont
    [Mapping=tex-text, Ligatures={Common}]
    {Linux Libertine O}

\newcommand{\court}[1]
    {\parbox[t]{12cm}{\uppercase{#1}}}
\newcommand{\fileno}[1]
    {\parbox[t]{3cm}{\raggedleft{#1}}\\[18pt]}
\newcommand{\party}[2]
    { \court{\textbf{#1}}
      \fileno{#2} }
\newcommand{\tram}[2]{
    \vspace{-24pt}
    \hrulefill
    \vspace{6pt}
    \begin{center}
        #1
    \end{center}
    \hrulefill
}
\newcommand{\heading}[1]{
    \textbf{#1}
}
\newcommand{\citn}[2]{
    \vspace{6pt}
    \emph{#1} #2.
}
\newcommand{\para}[2]{
    \begin{quote}
    [\textbf{#1}] #2
    \end{quote}
}


% no section numbering
\renewcommand{\thesection}{\hspace{-1em}}

% lists
\usepackage[shortlabels]{enumitem}
\setlist{leftmargin=1cm, align=left, resume}
\setlist[enumerate]{label*=\arabic*.}

\begin{document}

% no page numbers
\pagestyle{empty}

\court{In the Federal Court of Australia \\ Western Australia district registry \\ General Division}
\fileno{WAD 005 of 2013}
BETWEEN: \\[18pt]
\party{Friends of the North West Inc}{Applicant}
and \\[18pt]
\party{Minister for the Environment, Heritage and Water}{Respondent}

\tram{
    \uppercase{\textbf{
        Respondent's outline of submissions
    }}
            \\ (Junior Counsel) \\
    }

\section{Statement of facts}

\begin{enumerate}[1.]
\item
  Petro Energy Pty Ltd proposes to build and operate a floating LNG
  plant to process gas in the Selinka Gas Field, approximately 150 km
  off the coast of WA (\textbf{Proposal}). The Proposal is likely to
  emit noise and chemical pollution into the surrounding environment,
  which intersects a humpback whale migratory route.
\item
  Humpback whales are a `listed threatened species' under the
  \emph{Environment Protection and Biodiversity Conservation Act 1999}
  (Cth) (\textbf{\emph{EPBC Act}}). Section 18 of the \emph{EPBC Act}
  prohibits action that `has or will have a significant impact' on a
  listed threatened species without the Minister's approval.
\item
  Section 67 of the \emph{EPBC Act} provides that actions which are
  prohibited without the Minister's approval are `controlled actions.'
  On 10 June 2013, the Minister determined that the Proposal was a
  controlled action.
\item
  Section 87 of the \emph{EPBC Act} requires the Minister to choose one
  of several methods of assessing the environmental impact of a
  controlled action. On 10 June 2013, the Minister decided to assess the
  Proposal on referral information.
\item
  Section 93 of the \emph{EPBC Act} provides that within 30 days of the
  Minister deciding to assess a controlled action on referral
  information, the Department of Sustainability, Environment, Water,
  Population and Communities (\textbf{Department}) must:

  \begin{enumerate}
  \item
    publish a draft report containing a recommendation as to whether the
    controlled action should be approved (\textbf{Report}); and
  \item
    invite the public to provide comments in relation to the Report
    within 10 business days.
  \end{enumerate}

  The Department published a draft Report in respect of the Proposal on
  10 July 2013.
\item
  On 17 July 2013, the Applicant:

  \begin{enumerate}
  \item
    submitted several scientific studies concerning the environmental
    impact of floating LNG plants to the Department; and
  \item
    asked the Minister in writing to delay his decision on the Proposal
    until after the release of a new scientific study by the
    International Whaling Commission, due to be released in early August
    2013 (\textbf{Delay Request}).
  \end{enumerate}
\item
  On 25 July 2013, the Department finalised the Report and gave it to
  the Minister.
\item
  Section 130 of the \emph{EPBC Act} required the Minister to approve
  the Proposal within 20 business days after receiving the finalised
  Report, or ``such longer period as the Minister specifies in
  writing.''
\item
  On 30 July 2013, the Minister approved the Proposal
  (\textbf{Approval}) subject to conditions. The Minister's reasons for
  the Approval (\textbf{Reasons}) stated that the Minister:

  \begin{enumerate}
  \item
    had given ``strong consideration'' to the Commonwealth government's
    policy of ``streamlining'' environmental approval of offshore gas
    projects and ``cutting environmental green tape'' in order to ensure
    that the Australia offshore gas industry remained competitive and
    attractive to international investment (\textbf{Policy}); and
  \item
    refused the Delay Request as he considered that adequate time had
    been given for public comment.
  \end{enumerate}
\item
  The Applicant seeks judicial review of the Approval and the Minister's
  refusal of the Delay Request under the \emph{Administrative Decisions
  (Judicial Review) Act 1977} (\textbf{\emph{ADJR Act}}).
\end{enumerate}

\section{Submission 1: The Approval was not an improper exercise of
power within the meaning of s 5(2)(f) of the \emph{ADJR Act} (inflexible
application of policy).}

\begin{enumerate}[1.]
\item
  Section 5(2)(f) of the \emph{ADJR Act} provides that ``an exercise of
  a discretionary power in accordance with a rule or policy without
  regard to the merits of the particular case'' is improper.
\item
  A decision-maker is normally entitled to treat government policy as a
  relevant consideration in the absence of statutory provision to the
  contrary. The propriety of paying regard to policy considerations is
  particular evidence where the power is entrusted to a Minister of the
  Crown responsible to Parliament.

  \emph{Drake v Minister for Immigration and Ethnic Affairs} (1979) 46
  FLR 409, 420 \\(Bowen CJ, Deane J).
\item
  Although the Minister's power to approve the Proposal is
  discretionary, it is confined by the statement of relevant
  considerations in s 136 of the \emph{EPBC Act}. Section 136 is an
  exhaustive statement: s 136(5). However, it includes the broad
  category of ``economic and social matters'' as a mandatory
  consideration: s 136(1)(b).
\item
  The Policy is not specifically directed at any administrative power.
  Rather, it provides that the environmental approval decisions should
  be ``streamlined'' for economic reasons. The Minister was entitled to
  take the Policy into account when considering ``economic and social
  matters.''
\item
  In the absence of any statutory or contextual indication of the weight
  to be given to factors to which a decision-maker must have regard, it
  is generally for him or her to determine the appropriate weight to be
  given to them.

  \emph{Minister for Aboriginal Affairs v Peko-Wallsend Ltd} {[}1986{]}
  HCA 40; \\(1986) 162 CLR 24, 41 (Mason J).
\item
  There is nothing inherently wrong in the Minister pursuing a policy
  which:

  \begin{enumerate}
  \item
    is consistent with the statute under which the relevant power is
    conferred;
  \item
    does not preclude the Minister from taking into account relevant
    considerations; and
  \item
    does not require the Minister to take into account irrelevant
    considerations.
  \end{enumerate}

  \emph{NEAT Domestic Trading v AWB Ltd} (2003) 216 CLR 277, 289
  {[}24{]} (Gleeson CJ).
\item
  Although the Minister gave ``strong consideration'' to the Policy, he
  gave genuine consideration to the merits of the Proposal before
  approving it, as evidenced by the Minister's imposition of detailed
  conditions on the Proposal.
\end{enumerate}

\section{Submission 2: The Minister's refusal of the Delay Request was
not an improper exercise of power within the meaning of s 5(2)(g) of the
\emph{ADJR Act} (unreasonableness).}

\begin{enumerate}[1.]
\item
  Section 5(2)(g) of the \emph{ADJR Act} provides that ``an exercise of
  a power that is so unreasonable that no reasonable person could have
  so exercised the power'' is improper. Section 5(2)(g) reflects the
  common law notion of `\emph{Wednesbury} unreasonableness.'

  \emph{Associated Provincial Picture Houses Ltd v Wednesbury
  Corporation} {[}1948{]} 1 KB 223, \\ 230 (Lord Green MR).

  \emph{Minister for Aboriginal Affairs v Peko-Wallsend Ltd} {[}1986{]}
  HCA 40; \\(1986) 162 CLR 24, 41 (Mason J).
\item
  By requiring the Minister to make decisions within prescribed time
  periods, s 130 of the \emph{EPBC Act} reflects the legislature's
  intention, expressed in s 3(2)(d), that the environmental assessment
  and approval process should be ``efficient and timely.''
\item
  Section 130(1A) of the \emph{EPBC Act} grants the Minister a
  discretion to delay his decision for ``such longer period as the
  Minister specifies in writing.'' The discretion is unconfined by
  statutory criteria.
\item
  In \emph{Li}, the High Court affirmed the decision of the Full Court
  of the Federal Court granting judicial review of the Migration Review
  Tribunal's refusal to delay a decision. Gageler J discussed
  \emph{Wednesbury} unreasonableness, holding that:

  \begin{quote}
  Judicial determination of \emph{Wednesbury} unreasonableness in
  Australia has in practice been rare. Nothing in these reasons should
  be taken as encouragement to greater frequency. This is a rare case.
  \end{quote}

  \emph{Minister for Immigration and Citizenship v Li} {[}2013{]} HCA
  18, {[}113{]} (Gageler J).
\item
  Although Gageler J was satisfied that the Tribunal had acted
  unreasonably, its refusal to delay directly affected the Tribunal's
  ultimate decision. The delay was sought pending the review of a
  `skills assessment' which was unfavourable to the applicant.
  Ultimately, the review resulted in an assessment favouring the
  applicant. The Tribunal relied on the erroneous skills assessment in
  deciding the refuse the applicant a visa. In contrast, the Delay
  Request in this case was for the purpose of providing further
  information which the Minister might have regard to, not for the
  purpose of correcting information which the Minister was required to
  rely on.

  \emph{Minister for Immigration and Citizenship v Li} {[}2013{]} HCA
  18, {[}122{]} (Gageler J).
\item
  In \emph{Tarkine National Coalition}, this Court considered an
  application for judicial\\ review of a Ministerial decision under the
  \emph{EPBC Act}. Although the application was successful, the Court
  rejected the ground of review alleging \emph{Wednesbury}
  unreasonableness, noting Gageler J's observation.

  \emph{Tarkine National Coalition Incorporated v Minister for
  Sustainability, Environment, Water, Population and Communities}
  {[}2013{]} FCA 694, {[}84{]} (Marshall J).
\item
  It is not the function of the Court to substitute its own decision for
  that of the Minister. For the Court to intervene on the ground of
  unreasonableness would require ``something overwhelming.''

  \emph{Minister for Aboriginal Affairs v Peko-Wallsend Ltd} {[}1986{]}
  HCA 40; \\(1986) 162 CLR 24, 40-41 (Mason J).

  \emph{Tarkine National Coalition Incorporated v Minister for
  Sustainability, Environment, Water, Population and Communities}
  {[}2013{]} FCA 694, {[}85{]} (Marshall J).

  \emph{Associated Provincial Picture Houses Ltd v Wednesbury
  Corporation} {[}1948{]} 1 KB 223, \\ 230 (Lord Green MR).
\end{enumerate}

\raggedleft
\vfill
Dated this 21\textsuperscript{st} day of August 2013

\vspace{48pt}

\textbf{Scott Young} \\Junior Counsel for the Respondent

\end{document}
