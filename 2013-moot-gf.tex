\documentclass{../courtdoc-xelatex/courtdoc}

\begin{document}

\thispagestyle{empty} % suppress header on page 1

\court{In the Federal Court of Australia \\ Western Australia district registry \\ General Division}
\fileno{WAD 005 of 2013}
BETWEEN: \\[18pt]
\party{Friends of the North West Inc}{Applicant}
and \\[18pt]
\party{Minister for the Environment, Heritage and Water}{Respondent}

\tram{Respondent's outline of submissions \\ \textnormal{(J\lowercase{unior} C\lowercase{ounsel})}}


\heading{Statement of facts}
\enum{
    \item On 30 July 2013, the Minister approved the proposal (\textbf{Decision}). The
Minister's reasons for the Decision (\textbf{Reasons}) stated that: \enum{
        \item{in deciding to approve the taking of the action, the Minister had
given strong consideration to a recent Commonwealth government policy
announcement that it would 'streamline' environmental approval of offshore gas
projects and ‘cut environmental green tape' in order to ensure that the
Australia offshore gas industry remained competitive and attractive to
international investment;}
        \item{he had not delayed the decision in response to FNW’s letter, as
he considered that adequate time had been given for public comment in
compliance with the provisions of the EPBC Act.}
    }
}

\heading{Submission 1: The Minister did not make the Decision according to
a rule or policy without regard to the merits of the particular case.}
\enumnew{

\item{
    A lawful policy is normally a relevant consideration which a decision-maker
    is bound to take into account.

    \cite[passim]{drake}
    %\citn{Drake v Minister for Immigration and Ethnic Affairs}
    %{(1979) 24 ALR 577}
}

\item{
    In \emph{Drake}, the Court said:
    \para{420}{The propriety of paying regard to general policy
    considerations is most evident in a case such as the present where
    there are no specified statutory criteria for the exercise of the
    discretionary power and where the power is entrusted to a Minister
    of the Crown responsible to Parliament.}
    
The power is exercised by a Minister responsible to Parliament.

\item{
    NEAT Domestic Trading:
    \para{17}{These considerations do not preclude the person on whom the power is conferred from developing and applying a policy as to the approach which he will adopt in the generality of cases … But the position is different if the policy adopted is such as to preclude the person on whom the power is conferred from departing from the policy or from taking into account circumstances which are relevant to the particular case in relation to which the discretion is being exercised. If such an inflexible and invariable policy is adopted, both the policy and the decisions taken pursuant to it will be unlawful.}
}

\item{
    Section 136(1)(b) provides that the Minister ``must consider \ldots economic and social
    matters,'' so far as they they are ``are not inconsistent with any other requirement of this
    Subdivision.''}


\item{The Minister had appropriate regard to the merit of the case before making the Decision.}
}
}

\heading{Submission 2: The Minister's refusal to delay the decision was
reasonable in the circumstances.}

\enum{
    \item{Wednesbury unreasonableness.}
}

\end{document}
